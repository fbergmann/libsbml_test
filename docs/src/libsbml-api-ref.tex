\documentclass{sbmlmanual}

\newcommand{\libsbml}{\textsc{libsbml}}

\begin{document}

%=============================================================================
% Title page
%=============================================================================

\title{\textsc{libsbml} API Reference Manual}

\author{Ben Bornstein}

\authoremail{bornstei@cds.caltech.edu}

\address{The SBML Team\\
  Control and Dynamical Systems, MC 107-81\\
  California Institute of Technology, Pasadena, CA 91125, USA\\[2pt]
  {\url{http://www.sbml.org/}}}

\date{DRAFT\\[5pt]
  \today{}\\[10pt]%
  \hspace*{-1ex}\libsbml{} Version \input{../../VERSION.txt}%
}

\maketitlepage


%=============================================================================
\section{Introduction}
\label{sec:quickstart-unix}
%=============================================================================

This manual is a reference for the \libsbml{} application programming
interface (API).  \libsbml{} provides C and C++ APIs for reading, writing
and manipulating the Systems Biology Markup
Language~\citep[SBML;][]{hucka_2001b,hucka_2003,finney_2003b}.  Currently,
the library supports SBML Level~1 Version~1 and Version~2, and nearly all
of SBML Level~2 Version~1.  (The still-unimplemented parts of Level~2 are:
support for RDF, and support for MathML's \code{semantics},
\code{annotation} and \code{annotation-xml} elements.  These will be
implemented in the near future.)  For more information about SBML, please
see the references or visit \url{http://www.sbml.org/} on the Internet.

\libsbml{} is entirely open-source and all specifications and source code
are freely and publicly available.  This document explains the library API
in detail, but does not provide general information about \libsbml{}, its
use or its installation.  For that, please consult the
\emph{\libsbml{} Developer's Manual}~\citep{bornstein_2004}.


%=============================================================================
\section{API Reference}
%=============================================================================

\include{api/AlgebraicRule}
\include{api/AssignmentRule}
\include{api/ASTNode}
\include{api/Compartment}
\include{api/CompartmentVolumeRule}
\include{api/EventAssignment}
\include{api/Event}
\include{api/FormulaFormatter}
\include{api/FormulaParser}
\include{api/FormulaTokenizer}
\include{api/FunctionDefinition}
\include{api/KineticLaw}
\include{api/List}
\include{api/ListOf}
\include{api/MathMLDocument}
\include{api/MathMLReader}
\include{api/Model}
\include{api/ModifierSpeciesReference}
\include{api/ParameterRule}
\include{api/Parameter}
\include{api/ParseMessage}
\include{api/RateRule}
\include{api/Reaction}
\include{api/Rule}
\include{api/RuleType}
\include{api/SBase}
\include{api/SBMLDocument}
\include{api/SBMLReader}
\include{api/SBMLWriter}
\include{api/SimpleSpeciesReference}
\include{api/SpeciesConcentrationRule}
\include{api/SpeciesReference}
\include{api/Species}
\include{api/Stack}
\include{api/StringBuffer}
\include{api/UnitDefinition}
\include{api/UnitKind}
\include{api/Unit}
\include{api/util}


%=============================================================================
% References
%=============================================================================

\bibliographystyle{apalike}
\bibliography{libsbml}

%=============================================================================
% The end.
%=============================================================================

\end{document}
